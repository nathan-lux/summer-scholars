\documentclass[a4paper]{article}
\usepackage[skip=13pt]{parskip}
\usepackage{bm} 
\usepackage{graphicx} 
\usepackage{amsmath} 
\usepackage{amsfonts}
\usepackage{wrapfig} 
\usepackage{setspace} 
\usepackage{lipsum} 
\usepackage{cite}

\setstretch{1.25}
\setlength\parindent{0pt}

\begin{document}

    \section{\textbf{Introduction}}
    Astrophysics as a field is facing one of the most exciting times in history
    where the development of increasingly sophisticated technology has ushered
    in measurements with accuracy previously unknown to humankind. In
    particular, the launch and initialization of the James Web Space Telescope
    (JWST) in December 2021 and throughout early 2022 has been a revolutionary
    moment in modern astrophysics. Since then, JWST has provided physicists of
    various fields countless opportunities to verify or otherwise test the
    validity of modern cosmology to an even higher degree of accuracy than the
    previous generation of orbital telescopes has allowed. Among these theories
    we find the Standard Cosmological Model (SCM) (also called $\Lambda CDM$
    cosmological model or the Big Bang Theory) which has dominated the field of
    cosmology for decades; and for good reason.

    The SCM posits that the universe was once an incredibly dense sea of
    energy, too dense in fact for the formation of familiar matter to form.
    Accordingly it is proposed that this density’s associated heat fused the
    fundamental forces of nature into a larger grand unified force of nature.
    This dense sea of energy then underwent a period of inflation experiencing
    an exponential expansion of the universe. The universe would then slow it’s
    rate of inflation yet continue expanding allowing the matter to cool over
    time and eventually (over billions of years) allow structures like stars,
    galaxies, and all the other wonderful celestial bodies humanity has
    familiarized themselves with over the last several centuries.

    This is the model familiar to countless, it’s popularity even penetrating
    popular media. Thus the Big Bang Theory became a “household theory.” This,
    however, was not without reason. The SCM has an immense capacity to explain
    the universe as we see it and made predictions about the cosmos with
    unprecedented accuracy.
    \begin{center}
        [Will go on to discuss recent JWST Data and alternate model
        possibilties]
    \end{center}


\end{document}

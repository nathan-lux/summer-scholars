\documentclass[a4paper]{article}
\usepackage[skip=13pt]{parskip}
\usepackage{bm} 
\usepackage{graphicx} 
\usepackage{amsmath} 
\usepackage{amsfonts}
\usepackage{wrapfig} 
\usepackage{setspace} 
\usepackage{lipsum} 
\usepackage{cite}

\setstretch{1.25}
\setlength\parindent{0pt}

\begin{document}
\subsection{Average Temperature of CMB}
    
    The average temperature of the CMB is a critical aspect of the CMB as a
    phenomenon in the universe. This, along with the power spectrum characterize
    the primary features of the CMB, and thus, if the chronometric model is able to
    reproduce these aspects of the CMB, we can say that the model is, if nothing
    else, not falsified in light of modern astrophysical data. In order to
    determine these attributes from first principle, however, it is important to
    first get the bigger picture of how the CMB is accounted for within this model.
    
    Light in the chronometric cosmos is broadly separated into two categories. The
    first we call pristine light, and the second we call residual. We call the
    light that has taken fewer than one half-cycle about the cosmos pristine and
    the light which has passed the $\rho = \pi$ manifold distance residual. The CMB
    then is concerned primarily with the residual light in the universe. 
    
    Given the relative sparsity of matter in the universe, light of this category
    would be able to take many circuits about the universe. The infinite time for
    this high-dispersion radiation to accumulate would directly imply that it is
    qualitatively distributed in accordance to Planck’s law, and thus is in fact a
    black body spectrum.  
    
    \begin{center}
        [This will initiate the section on Plank law and that general
        calculation].  
    \end{center}
    
    It is worth noting before continuing that the origin of this light is
    unimportant for general considerations.  However for a more specific analysis
    we can easily consider the source of the pristine and residual  light in the
    universe to be the galaxies and other luminous matter in the universe. 
    
    To show that the residual light is characteristic of a black body spectrum, we
    need only point to the stochastic nature of of the emission and motion of
    galaxies and other luminous material.

    \begin{center}
        [This will end the Planck law calculation and start the extinction
        calculation]. 
    \end{center}

    \subsubsection{Methodology}
    
    The  methodology, then, for determining the average temperature of the black body
    spectrum in this model will be to utilize the residual light in the cosmos.
    The residual light will be considered explicitly as the light "left over"
    after traversing multiple half circuits of the cosmos, and is thus the
    light which has \textit{not} gone extinct.  

    With this, we will discover a unique relationship between it and the
    pristine light in the universe, and further find how, after several cycles,
    this light creates the black body which can be measured, observed, and
    used to calculate the average temperature of the CMB.

    This analysis will be quite general and is only intended to act as a first
    order approximation to the average temperature to the CMB. We will consider
    all galaxies to be \textit{approximately} the same in radius ($d$),
    luminosity ($L$), and have an explicit number count ($N$). Using these
    factors we will calculate on average the amounts of pristine and residual
    light in the universe, and what of that reaches the Earth which will give
    us qualitative information regarding the CMB in this model. We will then
    utilize that to determine a range of values for the average temperature of
    the CMB.

    \subsubsection{Pristine Light Calculation}

    To start, we will first attempt to calculate the amount of pristine light
    in the universe. The pristine light we will take (as is the case for this
    analysis in general) to be coming from galaxies in the universe.
    \subsubsection{Residual Light Calculation}

    We will now attempt to calculate the residual light. In this case, we need
    only consider the amount of light that arrives at the Earth, as this is the
    only light which we can factor into the chronometric model's explanation of
    the CMB. In order to calculate this, then, we can shift the question to be
    how much light is absorbed by galaxies in the universe en route to the
    Earth? Essentially, how much of the sky is taken up by galaxies.

    We consider the general volume of the $\mathbb{S}^3$ as a function of the
    manifold distance $\rho$.

    \begin{equation}
        V(\rho) = 
    \end{equation}
\end{document}


\documentclass[11pt, letterpaper]{article}
\usepackage[utf8]{inputenc}
\usepackage{amsmath}
\usepackage{amssymb}
\setlength{\parindent}{0pt}

\title{Homework 2/3}
\author{Nathan Burwig}

\begin{document}
    \maketitle

    \section*{Problem 1}
    Take a vector $x=x^1e_1+x^2e_2+...+x^ne_n$ with components $x^i$ in the
    basis ${e_i}$. Let ${e'_i}$ be a new basis and a transformation matrix A
    between ${e_i}$ and ${e'_i}$ given by a set of linear equations: \\
    
 
    \setlength{\leftskip}{3cm}
     $e'_1 = a_{11}e_1+a_{12}e_2+...+a_{1n}e_n$ \\
      .\\
      .\\
     $e'_n = a_{n1}e_1+a_{n2}e_2+...+a_{nn}e_n$ \\
     

    \setlength{\leftskip}{0cm}


     In the new basis, vector x can be written as
     $x'=x'^1e'_1+x'^2e'_2+...+x'^ne'_n$, it will have different components,
     but it is the same vector $x=x'$.\\

     Find how the components of vector transform under base change given by $A$
     - substitute for {$e'_i$} and find the transformation between $x_1,$
     $x_2...x_n$ and $x'_1,x'_2...x'_n$

     
     \section*{Problem 2}
     The notation is $x^k$ for components of a contravariant vector (tensor of
     rank 1 - tensor with 1 contravariant index and 0 covariant indices) and
     $x_k$ for a covariant vector.\\

     If base vectors transform like:\\

     \qquad$e_{k'}=A_{k'}^{i} e_i$, or $e'=Me$, or $e=M^{-1}e'$\\

     Then a contravariant vector (like a position vector) transforms like:\\

     \qquad$x_{k'}=A^i_{k'}x_i$, or $x' = Mx$\\

     in the same way as coefficents of a linear 1-form
     $f(x)\rightarrow\mathbb{R}$:\\

     \qquad $f_{k'}=A^i_{k'}f_i$, or $f'=Mf$\\

     Show explicitly, that the gradient of a scalar function taken with respect
     to a contravariant vector transforms like a covariant vector, and a
     derivative of a scalar function with respect to a covariant vector
     transforms transforms like a contravariant vector.\\

     \section*{Problem 3}
     Take an anti-symmetric bilinear form $f_{i,j}=-f_{i,j}$ in 3-dimensions -
     see the handout where it is shown how coefficients of a 1-form transform,
     and do the same for an anti-symmetric 2-form
     $f(a,b)=-f(b,a)\rightarrow\mathbb{R}$, where $a,b$ two vectors (a linear
     two-form is a function which takes two contravariant vectors as arguments
     and gives a number in a field over which the vector field is defined - in
     our case it is $\mathbb{R}$)\\

     Make the identification $g^1=f_{23}, g^2=f_{31}, g^3=f_{12}$\\

     Show that components $g^1$ transform under a base transformation {$e_k$}
     $\rightarrow$ {$e_{k'}$} given by $e_{k'}=A^i_{k'}e_i$, or $e'=Me$\\

     Like $g'=det(M)M^{-1}g$\\

     Notice that a covariant vector transforms like $x'=M^{-1}x$.\\

     What we call a vector product in 3D of two vectors $\vec a$ and $\vec b$,
     $\vec c = \vec a$ is really an object which is a result of identifying the
     coefficients of an anti-symmetric two form $f_{i,j}(a,b).a$ with
     $f^k=\epsilon_{ijk}f_{ij}=c$, where $\epsilon_{ijk}$ is the completely
     antisymmetric Levi-Civirta symbol (or tensor).
    
\end{document}
